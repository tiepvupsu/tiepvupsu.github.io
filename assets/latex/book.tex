%%%%%%%%%%%%%%%%%%%% book.tex %%%%%%%%%%%%%%%%%%%%%%%%%%%%%
%
% sample root file for the chapters of your "monograph"
%
% Use this file as a template for your own input.
%
%%%%%%%%%%%%%%%% Springer-Verlag %%%%%%%%%%%%%%%%%%%%%%%%%%


% RECOMMENDED %%%%%%%%%%%%%%%%%%%%%%%%%%%%%%%%%%%%%%%%%%%%%%%%%%%
\documentclass[envcountsame,envcountchap, openany]{mysvmono}
\usepackage[margin=1in]{geometry}
\usepackage[T5]{fontenc}

\usepackage[utf8]{inputenc}
\usepackage{amsmath}
\usepackage{amssymb}
\setlength{\parindent}{0em}
\setlength{\parskip}{1em}
% choose options for [] as required from the list
% in the Reference Guide, Sect. 2.2

\usepackage{makeidx}         % allows index generation
\usepackage{graphicx}        % standard LaTeX graphics tool
                             % when including figure files
\usepackage{multicol}        % used for the two-column index
\usepackage[bottom]{footmisc}% places footnotes at page bottom
% etc.
% see the list of further useful packages
% in the Reference Guide, Sects. 2.3, 3.1-3.3

\makeindex             % used for the subject index
                       % please use the style svind.ist with
                       % your makeindex program

\usepackage{cite,url,bm,hyperref}
\usepackage{hyperref}
\hypersetup{
    colorlinks=true,
    linkcolor=blue,
    filecolor=magenta,      
    urlcolor=blue,
}

%%%%%%%%%%%%%%%%%%%5
\usepackage{listings}
\usepackage{color}

%New colors defined below
\definecolor{codegreen}{rgb}{0,0.6,0}
\definecolor{codegray}{rgb}{0.5,0.5,0.5}
\definecolor{codepurple}{rgb}{0.58,0,0.82}
\definecolor{backcolour}{rgb}{0.95,0.95,0.92}


\usepackage{courier}

%Code listing style named "mystyle"
\lstdefinestyle{mystyle}{
  backgroundcolor=\color{backcolour},   commentstyle=\color{codegreen},
  keywordstyle=\color{magenta},
  stringstyle=\color{codepurple},
  basicstyle=\footnotesize,
  breakatwhitespace=false,         
  breaklines=true,                 
  captionpos=b,                    
  keepspaces=true,                 
  numbers=left,                    
  numbersep=5pt,                  
  showspaces=false,                
  showstringspaces=false,
  showtabs=false,                  
  tabsize=2, 
  basicstyle=\footnotesize\ttfamily,
  numberstyle=\tiny\color{white},
}

\lstset{basicstyle=\footnotesize\ttfamily,breaklines=true}

%"mystyle" code listing set
\lstset{style=mystyle}
%%%%%%%%%%%%%%%%%%%
\usepackage[font={normalsize}]{caption}
%%%%%%%%%%%%%%%%%%%%%%%%%%%%%%%%%%%%%%%%%%%%%%%%%%%%%%%%%%%%%%%%%%%%%

\begin{document}

\author{Vũ Hữu Tiệp}
\title{\bf Machine Learning cơ bản\\
{\small First Edition}}
% }
% \subtitle{-- Monograph --}
\maketitle

\frontmatter%%%%%%%%%%%%%%%%%%%%%%%%%%%%%%%%%%%%%%%%%%%%%%%%%%%%%%

% \include{dedic}
% \include{pref}

\tableofcontents


\mainmatter%%%%%%%%%%%%%%%%%%%%%%%%%%%%%%%%%%%%%%%%%%%%%%%%%%%%%%%
% \include{part}
% \include{chapter}
\include{1_introduce}
Có hai cách phổ biến phân nhóm các thuật toán Machine learning. Một là
dựa trên phương thức học (learning style), hai là dựa trên chức năng
(function) (của mỗi thuật toán).

\textbf{Trong trang này:}

\begin{itemize}
\tightlist
\item
  \protect\hyperlink{-phan-nhom-dua-tren-phuong-thuc-hoc}{1. Phân nhóm
  dựa trên phương thức học}

  \begin{itemize}
  \tightlist
  \item
    \protect\hyperlink{supervised-learning-hoc-co-giam-sat}{Supervised
    Learning (Học có giám sát)}

    \begin{itemize}
    \tightlist
    \item
      \protect\hyperlink{classification-phan-loai}{Classification (Phân
      loại)}
    \item
      \protect\hyperlink{regression-hoi-quy}{Regression (Hồi quy)}
    \end{itemize}
  \item
    \protect\hyperlink{unsupervised-learning-hoc-khong-giam-sat}{Unsupervised
    Learning (Học không giám sát)}

    \begin{itemize}
    \tightlist
    \item
      \protect\hyperlink{clustering-phan-nhom}{Clustering (phân nhóm)}
    \item
      \protect\hyperlink{association}{Association}
    \end{itemize}
  \item
    \protect\hyperlink{semi-supervised-learning-hoc-ban-giam-sat}{Semi-Supervised
    Learning (Học bán giám sát)}
  \item
    \protect\hyperlink{reinforcement-learning-hoc-cung-co}{Reinforcement
    Learning (Học Củng Cố)}
  \end{itemize}
\item
  \protect\hyperlink{-phan-nhom-dua-tren-chuc-nang}{2. Phân nhóm dựa
  trên chức năng}

  \begin{itemize}
  \tightlist
  \item
    \protect\hyperlink{regression-algorithms}{Regression Algorithms}
  \item
    \protect\hyperlink{classification-algorithms}{Classification
    Algorithms}
  \item
    \protect\hyperlink{instance-based-algorithms}{Instance-based
    Algorithms}
  \item
    \protect\hyperlink{regularization-algorithms}{Regularization
    Algorithms}
  \item
    \protect\hyperlink{bayesian-algorithms}{Bayesian Algorithms}
  \item
    \protect\hyperlink{clustering-algorithms}{Clustering Algorithms}
  \item
    \protect\hyperlink{artificial-neural-network-algorithms}{Artificial
    Neural Network Algorithms}
  \item
    \protect\hyperlink{dimensionality-reduction-algorithms}{Dimensionality
    Reduction Algorithms}
  \item
    \protect\hyperlink{ensemble-algorithms}{Ensemble Algorithms}
  \end{itemize}
\item
  \protect\hyperlink{-tai-lieu-tham-khao}{3. Tài liệu tham khảo}
\end{itemize}

\subsection{1. Phân nhóm dựa trên phương thức
học}\label{phuxe2n-nhuxf3m-dux1ef1a-truxean-phux1b0ux1a1ng-thux1ee9c-hux1ecdc}

Theo phương thức học, các thuật toán Machine Learning thường được chia
làm 4 nhóm: Supervise learning, Unsupervised learning, Semi-supervised
lerning và Reinforcement learning. \emph{Có một số cách phân nhóm không
có Semi-supervised learning hoặc Reinforcement learning.}

\subsubsection{Supervised Learning (Học có giám
sát)}\label{supervised-learning-hux1ecdc-cuxf3-giuxe1m-suxe1t}

Supervised learning là thuật toán dự đoán đầu ra (outcome) của một dữ
liệu mới (new input) dựa trên các cặp (\emph{input, outcome}) đã biết từ
trước. Cặp dữ liệu này còn được gọi là (\emph{data. label}), tức
(\emph{dữ liệu, nhãn}). Supervised learning là nhóm phổ biến nhất trong
các thuật toán Machine Learning.

Một cách toán học, Supervised learning là khi chúng ra có một tập hợp
biến đầu vào \textbackslash{}( \mathcal{X} =
\textbackslash{}\{\mathbf{x}\_1, \mathbf{x}\_2, \dots,
\mathbf{x}\_N\textbackslash{}\} \textbackslash{}) và một tập hợp nhãn
tương ứng \textbackslash{}( \mathcal{Y} =
\textbackslash{}\{\mathbf{y}\_1, \mathbf{y}\_2, \dots,
\mathbf{y}\_N\textbackslash{}\} \textbackslash{}), trong đó
\textbackslash{}( \mathbf{x}\_i, \mathbf{y}\_i \textbackslash{}) là các
vector. Các cặp dữ liệu biết trước \textbackslash{}( (\mathbf{x}\_i,
\mathbf{y}\_i) \in \mathcal{X} \times \mathcal{Y} \textbackslash{}) được
gọi là tập \emph{training data} (dữ liệu huấn luyện). Từ tập traing data
này, chúng ta cần tạo ra một hàm số ánh xạ mỗi phần tử từ tập
\textbackslash{}(\mathcal{X}\textbackslash{}) sang một phần tử (xấp xỉ)
tương ứng của tập \textbackslash{}(\mathcal{Y}\textbackslash{}):

\textbackslash{}{[} \mathbf{y}\_i \approx f(\mathbf{x}\_i),
\textasciitilde{}\textasciitilde{} \forall i = 1, 2, \dots,
N\textbackslash{}{]} Mục đích là xấp xỉ hàm số
\textbackslash{}(f\textbackslash{}) thật tốt để khi có một dữ liệu
\textbackslash{}(\mathbf{x}\textbackslash{}) mới, chúng ta có thể tính
được nhãn tương ứng của nó \textbackslash{}( \mathbf{y} = f(\mathbf{x})
\textbackslash{}).

\textbf{Ví dụ 1:} trong nhận dạng chữ viết tay, ta có ảnh của hàng nghìn
ví dụ của mỗi chữ số được viết bởi nhiều người khác nhau. Chúng ta đưa
các bức ảnh này vào trong một thuật toán và chỉ cho nó biết mỗi bức ảnh
tương ứng với chữ số nào. Sau khi thuật toán tạo ra (sau khi \emph{học})
một mô hình, tức một hàm số mà đầu vào là một bức ảnh và đầu ra là một
chữ số, khi nhận được một bức ảnh mới mà mô hình \textbf{chưa nhìn thấy
bao giờ}, nó sẽ dự đoán bức ảnh đó chứa chữ số nào.

\begin{verbatim}
<img src="http://www.rubylab.io/img/mnist.png" width = "600">
<!-- <img src="/assets/rl/mdp.png" height="206"> -->
\end{verbatim}

MNIST: bộ cơ sở dữ liệu của chữ số viết tay. (Nguồn: Simple Neural
Network implementation in Ruby)

Ví dụ này khá giống với cách học của con người khi còn nhỏ. Ta đưa bảng
chữ cái cho một đứa trẻ và chỉ cho chúng đây là chữ A, đây là chữ B. Sau
một vài lần được dạy thì trẻ có thể nhận biết được đâu là chữ A, đâu là
chữ B trong một cuốn sách mà chúng chưa nhìn thấy bao giờ.

\textbf{Ví dụ 2:} Thuật toán dò các khuôn mặt trong một bức ảnh đã được
phát triển từ rất lâu. Thời gian đầu, facebook sử dụng thuật toán này để
chỉ ra các khuôn mặt trong một bức ảnh và yêu cầu người dùng \emph{tag
friends} - tức gán nhãn cho mỗi khuôn mặt. Số lượng cặp dữ liệu
(\emph{khuôn mặt, tên người}) càng lớn, độ chính xác ở những lần tự động
\emph{tag} tiếp theo sẽ càng lớn.

\textbf{Ví dụ 3:} Bản thân thuật toán dò tìm các khuôn mặt trong 1 bức
ảnh cũng là một thuật toán Supervised learning với training data (dữ
liệu học) là hàng ngàn cặp (\emph{ảnh, mặt người}) và (\emph{ảnh, không
phải mặt người}) được đưa vào. Chú ý là dữ liệu này chỉ phân biệt
\emph{mặt người} và \emph{không phải mặt ngưòi} mà không phân biệt khuôn
mặt của những người khác nhau.

Thuật toán supervised learning còn được tiếp tục chia nhỏ ra thành hai
loại chính:

\paragraph{Classification (Phân
loại)}\label{classification-phuxe2n-loux1ea1i}

Một bài toán được gọi là \emph{classification} nếu các \emph{label} của
\emph{input data} được chia thành một số hữu hạn nhóm. Ví dụ: Gmail xác
định xem một email có phải là spam hay không; các hãng tín dụng xác định
xem một khách hàng có khả năng thanh toán nợ hay không. Ba ví dụ phía
trên được chia vào loại này.

\paragraph{Regression (Hồi quy)}\label{regression-hux1ed3i-quy}

(tiếng Việt dịch là \emph{Hồi quy}, tôi không thích cách dịch này vì bản
thân không hiểu nó nghĩa là gì)

Nếu \emph{label} không được chia thành các nhóm mà là một giá trị thực
cụ thể. Ví dụ: một căn nhà rộng \textbackslash{}(x \textasciitilde{}
\text{m}\^{}2\textbackslash{}), có \textbackslash{}(y\textbackslash{})
phòng ngủ và cách trung tâm thành phố
\textbackslash{}(z\textasciitilde{} \text{km}\textbackslash{}) sẽ có giá
là bao nhiêu?

Gần đây \href{http://how-old.net/}{Microsoft có một ứng dụng dự đoán
giới tính và tuổi dựa trên khuôn mặt}. Phần dự đoán giới tính có thể coi
là thuật toán \textbf{Classification}, phần dự đoán tuổi có thể coi là
thuật toán \textbf{Regression}. \emph{Chú ý rằng phần dự đoán tuổi cũng
có thể coi là \textbf{Classification} nếu ta coi tuổi là một số nguyên
dương không lớn hơn 150, chúng ta sẽ có 150 class (lớp) khác nhau.}

\subsubsection{Unsupervised Learning (Học không giám
sát)}\label{unsupervised-learning-hux1ecdc-khuxf4ng-giuxe1m-suxe1t}

Trong thuật toán này, chúng ta không biết được \emph{outcome} hay
\emph{nhãn} mà chỉ có dữ liệu đầu vào. Thuật toán unsupervised learning
sẽ dựa vào cấu trúc của dữ liệu để thực hiện một công việc nào đó, ví dụ
như phân nhóm (clustering) hoặc giảm số chiều của dữ liệu (dimention
reduction) để thuận tiện trong việc lưu trữ và tính toán.

Một cách toán học, Unsupervised learning là khi chúng ta chỉ có dữ liệu
vào \textbackslash{}(\mathcal{X} \textbackslash{}) mà không biết
\emph{nhãn} \textbackslash{}(\mathcal{Y}\textbackslash{}) tương ứng.

Những thuật toán loại này được gọi là Unsupervised learning vì không
giống như Supervised learning, chúng ta không biết câu trả lời chính xác
cho mỗi dữ liệu đầu vào. Giống như khi ta học, không có thầy cô giáo nào
chỉ cho ta biết đó là chữ A hay chữ B. Cụm \emph{không giám sát} được
đặt tên theo nghĩa này.

Các bài toán Unsupervised learning được tiếp tục chia nhỏ thành hai
loại:

\paragraph{Clustering (phân nhóm)}\label{clustering-phuxe2n-nhuxf3m}

Một bài toán phân nhóm toàn bộ dữ liệu
\textbackslash{}(\mathcal{X}\textbackslash{}) thành các nhóm nhỏ dựa
trên sự liên quan giữa các dữ liệu trong mỗi nhóm. Ví dụ: phân nhóm
khách hàng dựa trên hành vi mua hàng. Điều này cũng giống như việc ta
đưa cho một đứa trẻ rất nhiều mảnh ghép với các hình thù và màu sắc khác
nhau, ví dụ tam giác, vuông, tròn với màu xanh và đỏ, sau đó yêu cẩu trẻ
phân chúng thành từng nhóm. Mặc dù không cho trẻ biết mảnh nào tương ứng
với hình nào hoặc màu nào, nhiều khả năng chúng vẫn có thể phân loại các
mảnh ghép theo màu hoặc hình dạng.

\hypertarget{association}{\paragraph{Association}\label{association}}

Là bài toán khi chúng ta muốn khám phá ra một quy luật dựa trên nhiều dữ
liệu cho trước. Ví dụ: những khách hàng nam mua quần áo thường có xu
hướng mua thêm đồng hồ hoặc thắt lưng; những khán giả xem phim Spider
Man thường có xu hướng xem thêm phim Bat Man, dựa vào đó tạo ra một hệ
thống gợi ý khách hàng (Recommendation System), thúc đẩy nhu cầu mua
sắm.

\subsubsection{Semi-Supervised Learning (Học bán giám
sát)}\label{semi-supervised-learning-hux1ecdc-buxe1n-giuxe1m-suxe1t}

Các bài toán khi chúng ta có một lượng lớn dữ liệu
\textbackslash{}(\mathcal{X}\textbackslash{}) nhưng chỉ một phần trong
chúng được gán nhãn được gọi là Semi-Supervised Learning. Những bài toán
thuộc nhóm này nằm giữa hai nhóm được nêu bên trên.

Một ví dụ điển hình của nhóm này là chỉ có một phần ảnh hoặc văn bản
được gán nhãn (ví dụ bức ảnh về người, động vật hoặc các văn bản khoa
học, chính trị) và phần lớn các bức ảnh/văn bản khác chưa được gán nhãn
được thu thập từ internet. Thực tế cho thấy rất nhiều các bài toàn
Machine Learning thuộc vào nhóm này vì việc thu thập dữ liệu có nhãn tốn
rất nhiều thời gian và có chi phí cao. Rất nhiều loại dữ liệu thậm chí
cần phải có chuyên gia mới gán nhãn được (ảnh y học chẳng hạn). Ngược
lại, dữ liệu chưa có nhãn có thể được thu thập với chi phí thấp từ
internet.

\subsubsection{Reinforcement Learning (Học Củng
Cố)}\label{reinforcement-learning-hux1ecdc-cux1ee7ng-cux1ed1}

Reinforcement learning là các bài toán giúp cho một hệ thống tự động xác
định hành vi dựa trên hoàn cảnh để đạt được lợi ích cao nhất (maximizing
the performance). Hiện tại, Reinforcement learning chủ yếu được áp dụng
vào Lý Thuyết Trò Chơi (Game Theory), các thuật toán cần xác định nưóc
đi tiếp theo để đạt được điểm số cao nhất.

AlphaGo chơi cờ vây với Lee Sedol. AlphaGo là một ví dụ của
Reinforcement learning. (Nguồn: AlphaGo AI Defeats Sedol Again, With
`Near Perfect Game')

\textbf{Ví dụ 1:}
\href{https://gogameguru.com/tag/deepmind-alphago-lee-sedol/}{AlphaGo
gần đây nổi tiếng với việc chơi cờ vây thắng cả con người}.
\href{https://www.tastehit.com/blog/google-deepmind-alphago-how-it-works/}{Cờ
vây được xem là có độ phức tạp cực kỳ cao} với tổng số nước đi là xấp xỉ
\textbackslash{}(10\^{}\{761\} \textbackslash{}), so với cờ vua là
\textbackslash{}(10\^{}\{120\} \textbackslash{}) và tổng số nguyên tử
trong toàn vũ trụ là khoảng
\textbackslash{}(10\^{}\{80\}\textbackslash{})!! Vì vậy, thuật toán phải
chọn ra 1 nước đi tối ưu trong số hàng nhiều tỉ tỉ lựa chọn, và tất
nhiên, không thể áp dụng thuật toán tương tự như
\href{https://en.wikipedia.org/wiki/Deep_Blue_(chess_computer)}{IBM Deep
Blue} (IBM Deep Blue đã thắng con người trong môn cờ vua 20 năm trước).
Về cơ bản, AlphaGo bao gồm các thuật toán thuộc cả Supervised learning
và Reinforcement learning. Trong phần Supervised learning, dữ liệu từ
các ván cờ do con người chơi với nhau được đưa vào để huấn luyện. Tuy
nhiên, mục đích cuối cùng của AlphaGo không phải là chơi như con người
mà phải thậm chí thắng cả con người. Vì vậy, sau khi \emph{học} xong các
ván cờ của con người, AlphaGo tự chơi với chính nó với hàng triệu ván
chơi để tìm ra các nước đi mới tối ưu hơn. Thuật toán trong phần tự chơi
này được xếp vào loại Reinforcement learning. (Xem thêm tại
\href{https://www.tastehit.com/blog/google-deepmind-alphago-how-it-works/}{Google
DeepMind's AlphaGo: How it works}).

\textbf{Ví dụ 2:}
\href{https://www.youtube.com/watch?v=qv6UVOQ0F44}{Huấn luyện cho máy
tính chơi game Mario}. Đây là một chương trình thú vị dạy máy tính chơi
game Mario. Game này đơn giản hơn cờ vây vì tại một thời điểm, người
chơi chỉ phải bấm một số lượng nhỏ các nút (di chuyển, nhảy, bắn đạn)
hoặc không cần bấm nút nào. Đồng thời, phản ứng của máy cũng đơn giản
hơn và lặp lại ở mỗi lần chơi (tại thời điểm cụ thể sẽ xuất hiện một
chướng ngại vật cố định ở một vị trí cố định). Đầu vào của thuật toán là
sơ đồ của màn hình tại thời điểm hiện tại, nhiệm vụ của thuật toán là
với đầu vào đó, tổ hợp phím nào nên được bấm. Việc huấn luyện này được
dựa trên điểm số cho việc di chuyển được bao xa trong thời gian bao lâu
trong game, càng xa và càng nhanh thì được điểm thưởng càng cao (điểm
thưởng này không phải là điểm của trò chơi mà là điểm do chính người lập
trình tạo ra). Thông qua huấn luyện, thuật toán sẽ tìm ra một cách tối
ưu để tối đa số điểm trên, qua đó đạt được mục đích cuối cùng là cứu
công chúa.

Huấn luyện cho máy tính chơi game Mario

\subsection{2. Phân nhóm dựa trên chức
năng}\label{phuxe2n-nhuxf3m-dux1ef1a-truxean-chux1ee9c-nux103ng}

Có một cách phân nhóm thứ hai dựa trên chức năng của các thuật toán.
Trong phần này, tôi xin chỉ liệt kê các thuật toán. Thông tin cụ thể sẽ
được trình bày trong các bài viết khác tại blog này. Trong quá trình
viết, tôi có thể sẽ thêm bớt một số thuật toán.

\hypertarget{regression-algorithms}{\subsubsection{Regression
Algorithms}\label{regression-algorithms}}

\begin{enumerate}
\def\labelenumi{\arabic{enumi}.}
\tightlist
\item
  \href{/2016/12/28/linearregression/}{Linear Regression}
\item
  \href{/2017/01/27/logisticregression/\#sigmoid-function}{Logistic
  Regression}
\item
  Stepwise Regression
\end{enumerate}

\hypertarget{classification-algorithms}{\subsubsection{Classification
Algorithms}\label{classification-algorithms}}

\begin{enumerate}
\def\labelenumi{\arabic{enumi}.}
\tightlist
\item
  Linear Classifier
\item
  Support Vector Machine (SVM)
\item
  Kernel SVM
\item
  Sparse Represntation-based classification (SRC)
\end{enumerate}

\hypertarget{instance-based-algorithms}{\subsubsection{Instance-based
Algorithms}\label{instance-based-algorithms}}

\begin{enumerate}
\def\labelenumi{\arabic{enumi}.}
\tightlist
\item
  \href{/2017/01/08/knn/}{k-Nearest Neighbor (kNN)}
\item
  Learnin Vector Quantization (LVQ)
\end{enumerate}

\hypertarget{regularization-algorithms}{\subsubsection{Regularization
Algorithms}\label{regularization-algorithms}}

\begin{enumerate}
\def\labelenumi{\arabic{enumi}.}
\tightlist
\item
  Ridge Regression
\item
  Least Absolute Shringkage and Selection Operator (LASSO)
\item
  Least-Angle Regression (LARS)
\end{enumerate}

\hypertarget{bayesian-algorithms}{\subsubsection{Bayesian
Algorithms}\label{bayesian-algorithms}}

\begin{enumerate}
\def\labelenumi{\arabic{enumi}.}
\tightlist
\item
  Naive Bayes
\item
  Gausian Naive Bayes
\end{enumerate}

\hypertarget{clustering-algorithms}{\subsubsection{Clustering
Algorithms}\label{clustering-algorithms}}

\begin{enumerate}
\def\labelenumi{\arabic{enumi}.}
\tightlist
\item
  \href{/2017/01/01/kmeans/}{k-Means clustering}
\item
  k-Medians
\item
  Expectation Maximisation (EM)
\end{enumerate}

\hypertarget{artificial-neural-network-algorithms}{\subsubsection{Artificial
Neural Network Algorithms}\label{artificial-neural-network-algorithms}}

\begin{enumerate}
\def\labelenumi{\arabic{enumi}.}
\tightlist
\item
  \href{/2017/01/21/perceptron/}{Perceptron}
\item
  \href{/2017/02/17/softmax/}{Softmax Regression}
\item
  \href{/2017/02/24/mlp/}{Multi-layer Perceptron}
\item
  \href{/2017/02/24/mlp/\#-backpropagation}{Back-Propagation}
\end{enumerate}

\hypertarget{dimensionality-reduction-algorithms}{\subsubsection{Dimensionality
Reduction Algorithms}\label{dimensionality-reduction-algorithms}}

\begin{enumerate}
\def\labelenumi{\arabic{enumi}.}
\tightlist
\item
  Principal Component Analysis (PCA)
\item
  Linear Discriminant Analysis (LDA)
\end{enumerate}

\hypertarget{ensemble-algorithms}{\subsubsection{Ensemble
Algorithms}\label{ensemble-algorithms}}

\begin{enumerate}
\def\labelenumi{\arabic{enumi}.}
\tightlist
\item
  Boosting
\item
  AdaBoost
\item
  Random Forest
\end{enumerate}

Và còn rất nhiều các thuật toán khác.

\subsection{3. Tài liệu tham
khảo}\label{tuxe0i-liux1ec7u-tham-khux1ea3o}

\begin{enumerate}
\def\labelenumi{\arabic{enumi}.}
\item
  \href{http://machinelearningmastery.com/a-tour-of-machine-learning-algorithms/}{A
  Tour of Machine Learning Algorithms}
\item
  \href{https://ongxuanhong.wordpress.com/2015/10/22/diem-qua-cac-thuat-toan-machine-learning-hien-dai/}{Điểm
  qua các thuật toán Machine Learning hiện đại}
\end{enumerate}

%!TEX root = book.tex
\chapter{Linear Regression}

\index{Linear Regression}
 
Trong bài này, tôi sẽ giới thiệu một trong những thuật toán cơ bản nhất (và đơn giản nhất) của Machine Learning. Đây là một thuật toán \textit{Supervised learning} có tên {\textbf{Linear Regression}} (Hồi Quy Tuyến Tính). Bài toán này đôi khi được gọi là \textit{Linear Fitting} (trong thống kê) hoặc \textit{Linear Least Square}. 

\section{Giới thiệu}
 
Quay lại \href{http://machinelearningcoban.com/2016/12/27/categories/#regression-hoi-quy}{ví dụ đơn giản được nêu trong bài trước}: một căn nhà rộng $x_1 ~ \text{m}^2$, có $x_2$ phòng ngủ và cách trung tâm thành phố $x_3~ \text{km}$ có giá là bao nhiêu. Giả sử chúng ta đã có số liệu thống kê từ 1000 căn nhà trong thành phố đó, liệu rằng khi có một căn nhà mới với các thông số về diện tích, số phòng ngủ và khoảng cách tới trung tâm, chúng ta có thể dự đoán được giá của căn nhà đó không? Nếu có thì hàm dự đoán $y = f(\mathbf{x}) $ sẽ có dạng như thế nào. Ở đây $\mathbf{x} = [x_1, x_2, x_3] $ là một vector hàng chứa thông tin \textit{input}, $y$ là một số vô hướng (scalar) biểu diễn \textit{output} (tức giá của căn nhà trong ví dụ này). 
 
\textbf{Lưu ý về ký hiệu toán học:} \textit{trong các bài viết của tôi, các số vô hướng được biểu diễn bởi các chữ cái viết ở dạng không in đậm, có thể viết hoa, ví dụ $x_1, N, y, k$. Các vector được biểu diễn bằng các chữ cái thường in đậm, ví dụ $\mathbf{y}, \mathbf{x}_1 $. Các ma trận được biểu diễn bởi các chữ viết hoa in đậm, ví dụ $\mathbf{X, Y, W} $.} 
 
Một cách đơn giản nhất, chúng ta có thể thấy rằng: i) diện tích nhà càng lớn thì giá nhà càng cao; ii) số lượng phòng ngủ càng lớn thì giá nhà càng cao; iii) càng xa trung tâm thì giá nhà càng giảm. Một hàm số đơn giản nhất có thể mô tả mối quan hệ giữa giá nhà và 3 đại lượng đầu vào là:  
\begin{equation}
\label{eqn:linearregression}
y \approx   \hat{y} = f(\mathbf{x}) =w_1 x_1 + w_2 x_2 + w_3 x_3 + w_0 
\end{equation}
trong đó, $w_1, w_2, w_3, w_0$ là các hằng số,  $w_0$ còn được gọi là bias. Mối quan hệ $y \approx f(\mathbf{x})$ bên trên là một mối quan hệ tuyến tính (linear). Bài toán chúng ta đang làm là một bài toán thuộc loại regression. Bài toán đi tìm các hệ số tối ưu $ \\{w_1, w_2, w_3, w_0 \\}$ chính vì vậy được gọi là bài toán Linear Regression.  
 
\textbf{Chú ý 1:} $y$ là giá trị thực của \textit{outcome} (dựa trên số liệu thống kê chúng ta có trong tập \textit{training data}), trong khi $\hat{y}$ là giá trị mà mô hình Linear Regression dự đoán được. Nhìn chung, $y$ và $\hat{y}$ là hai giá trị khác nhau do có sai số mô hình, tuy nhiên, chúng ta mong muốn rằng sự khác nhau này rất nhỏ. 
 
\textbf{Chú ý 2:} \textit{Linear} hay \textit{tuyến tính} hiểu một cách đơn giản là \textit{thẳng, phẳng}. Trong không gian hai chiều, một hàm số được gọi là \textit{tuyến tính} nếu đồ thị của nó có dạng một \textit{đường thẳng}. Trong không gian ba chiều, một hàm số được goi là \textit{tuyến tính} nếu đồ thị của nó có dạng một \textit{mặt phẳng}. Trong không gian nhiều hơn 3 chiều, khái niệm \textit{mặt phẳng} không còn phù hợp nữa, thay vào đó, một khái niệm khác ra đời được gọi là \textit{siêu mặt phẳng} (\textit{hyperplane}). Các hàm số tuyến tính là các hàm đơn giản nhất, vì chúng thuận tiện trong việc hình dung và tính toán. Chúng ta sẽ được thấy trong các bài viết sau, \textit{tuyến tính} rất quan trọng và hữu ích trong các bài toán Machine Learning. Kinh nghiệm cá nhân tôi cho thấy, trước khi hiểu được các thuật toán \textit{phi tuyến} (non-linear, không phẳng), chúng ta cần nắm vững các kỹ thuật cho các mô hình \textit{tuyến tính}. 
 
 
 
 
 
\section{Phân tích toán học}
 
 
 
 
\subsection{Dạng của Linear Regression }
 
Trong phương trình $(1)$ phía trên, nếu chúng ta đặt $\mathbf{w} = [w_0, w_1, w_2, w_3]^T = $ là vector (cột) hệ số cần phải tối ưu và $\mathbf{\bar{x}} = [1, x_1, x_2, x_3]$ (đọc là \textit{x bar} trong tiếng Anh) là vector (hàng) dữ liệu đầu vào \textit{mở rộng}. Số $1$ ở đầu được thêm vào để phép tính đơn giản hơn và thuận tiện cho việc tính toán. Khi đó, phương trình (1) có thể được viết lại dưới dạng: 
$$y \approx \mathbf{\bar{x}}\mathbf{w} = \hat{y}$$ 
 
Chú ý rằng $\mathbf{\bar{x}}$ là một vector hàng. (\href{http://machinelearningcoban.com/math/#luu-y-ve-ky-hieu}{Xem thêm về ký hiệu vector hàng và cột tại đây}) 
 
 
 
\subsection{Sai số dự đoán }
 
Chúng ta mong muốn rằng sự sai khác $e$ giữa giá trị thực $y$ và giá trị dự đoán $\hat{y}$ (đọc là \textit{y hat} trong tiếng Anh) là nhỏ nhất. Nói cách khác, chúng ta muốn giá trị sau đây càng nhỏ càng tốt:  
 \begin{equation*}
\frac{1}{2}e^2 = \frac{1}{2}(y - \hat{y})^2 = \frac{1}{2}(y - \mathbf{\bar{x}}\mathbf{w})^2 
\end{equation*}
 trong đó hệ số $\frac{1}{2} $ (\textit{lại}) là để thuận tiện cho việc tính toán (khi tính đạo hàm thì số $\frac{1}{2} $ sẽ bị triệt tiêu). Chúng ta cần $e^2$ vì $e = y - \hat{y} $ có thể là một số âm, việc nói $e$ nhỏ nhất sẽ không đúng vì khi $e = - \infty$ là rất nhỏ nhưng sự sai lệch là rất lớn. Bạn đọc có thể tự đặt câu hỏi: \textbf{tại sao không dùng trị tuyệt đối $ \|e\| $ mà lại dùng bình phương $e^2$ ở đây?} Câu trả lời sẽ có ở phần sau.  
 
 
 
 
 
 
\subsection{Hàm mất mát}
 
Điều tương tự xảy ra với tất cả các cặp \textit{(input, outcome)} $ (\mathbf{x}_i, y_i), i = 1, 2, \dots, N $, với $N$ là số lượng dữ liệu quan sát được. Điều chúng ta muốn, tổng sai số là nhỏ nhất, tương đương với việc tìm $ \mathbf{w} $ để hàm số sau đạt giá trị nhỏ nhất: 
\begin{equation}
\label{eqn:linearregression_loss} 
 \mathcal{L}(\mathbf{w}) = \frac{1}{2}\sum_{i=1}^N (y_i - \mathbf{\bar{x}_i}\mathbf{w})^2
\end{equation}  
 
Hàm số $\mathcal{L}(\mathbf{w}) $ được gọi là \textbf{hàm mất mát} (loss function) của bài toán Linear Regression. Chúng ta luôn mong muốn rằng sự mất mát (sai số) là nhỏ nhất, điều đó đồng nghĩa với việc  tìm vector hệ số $ \mathbf{w} $  sao cho  
giá trị của hàm mất mát này càng nhỏ càng tốt. Giá trị của $\mathbf{w}$ làm cho hàm mất mát đạt giá trị nhỏ nhất được gọi là \textit{điểm tối ưu} (optimal point), ký hiệu: 
 
$$ \mathbf{w}^* = \arg\min_{\mathbf{w}} \mathcal{L}(\mathbf{w})  $$  
 
Trước khi đi tìm lời giải, chúng ta đơn giản hóa phép toán trong phương trình hàm mất mát \eqref{eqn:linearregression_loss}. Đặt $\mathbf{y} = [y_1; y_2; \dots; y_N]$ là một vector cột chứa tất cả các \textit{output} của \textit{training data}; $ \mathbf{\bar{X}} = [\mathbf{\bar{x}}_1; \mathbf{\bar{x}}_2; \dots; \mathbf{\bar{x}}_N ] $ là ma trận dữ liệu đầu vào (mở rộng) mà mỗi hàng của nó là một điểm dữ liệu. Khi đó hàm số mất mát $\mathcal{L}(\mathbf{w})$ được viết dưới dạng ma trận đơn giản hơn: 
\begin{equation}
\label{eqn:linearregression_loss_mat}
\mathcal{L}(\mathbf{w})  
= \frac{1}{2}\sum_{i=1}^N (y_i - \mathbf{\bar{x}}_i\mathbf{w})^2 
= \frac{1}{2} \|\mathbf{y} - \mathbf{\bar{X}}\mathbf{w} \|_2^2  
\end{equation} 
với $ \| \mathbf{z} \|_2 $ là Euclidean norm (chuẩn Euclid, hay khoảng cách Euclid), nói cách khác $ \| \mathbf{z} \|_2^2 $ là tổng của bình phương mỗi phần tử của vector $\mathbf{z}$. Tới đây, ta đã có một dạng đơn giản của hàm mất mát được viết như phương trình \eqref{eqn:linearregression_loss_mat}. 
 
 
 
 
\subsection{Nghiệm cho bài toán Linear Regression}
 
\textbf{Cách phổ biến nhất để tìm nghiệm cho một bài toán tối ưu (chúng ta đã biết từ khi học cấp 3) là giải phương trình đạo hàm (gradient) bằng 0!} Tất nhiên đó là khi việc tính đạo hàm và việc giải phương trình đạo hàm bằng 0 không quá phức tạp. Thật may mắn, với các mô hình tuyến tính, hai việc này là khả thi.  
 
Đạo hàm theo $\mathbf{w} $ của hàm mất mát là:  
$$ 
\frac{\partial{\mathcal{L}(\mathbf{w})}}{\partial{\mathbf{w}}}  
= \mathbf{\bar{X}}^T(\mathbf{\bar{X}}\mathbf{w} - \mathbf{y})  
$$ 
 
Các bạn có thể tham khảo bảng đạo hàm theo vector hoặc ma trận của một hàm số trong \href{https://ccrma.stanford.edu/~dattorro/matrixcalc.pdf}{mục D.2 của tài liệu này}. \textit{Đến đây tôi xin quay lại câu hỏi ở phần Sai số dự đoán phía trên về việc tại sao không dùng trị tuyệt đối mà lại dùng bình phương. Câu trả lời là hàm bình phương có đạo hàm tại mọi nơi, trong khi hàm trị tuyệt đối thì không (đạo hàm không xác định tại 0)}. 
 
Phương trình đạo hàm bằng 0 tương đương với:  
\begin{equation}
\label{eqn:linearregression_gradient}
\mathbf{\bar{X}}^T\mathbf{\bar{X}}\mathbf{w} = \mathbf{\bar{X}}^T\mathbf{y} \triangleq \mathbf{b}  
\end{equation}
(ký hiệu $\mathbf{\bar{X}}^T\mathbf{y} \triangleq \mathbf{b} $ nghĩa là \textit{đặt} $\mathbf{\bar{X}}^T\mathbf{y}$ \textit{bằng} $\mathbf{b}$ ). 
 
Nếu ma trận vuông $ \mathbf{A} \triangleq \mathbf{\bar{X}}^T\mathbf{\bar{X}}$ khả nghịch (non-singular hay inversable) thì phương trình $(4)$ có nghiệm duy nhất: $ \mathbf{w} = \mathbf{A}^{-1}\mathbf{b}  $. 
 
\index{pseudo inverse}
Vậy nếu ma trận $\mathbf{A} $ không khả nghịch (có định thức bằng 0) thì sao? Nếu các bạn vẫn nhớ các kiến thức về hệ phương trình tuyến tính, trong trường hợp này thì hoặc phương trinh \eqref{eqn:linearregression_gradient} vô nghiệm, hoặc là nó có vô số nghiệm. Khi đó, chúng ta sử dụng khái niệm \href{https://vi.wikipedia.org/wiki/Giả_nghịch_đảo_Moore–Penrose}{\textit{giả nghịch đảo}} $ \mathbf{A}^{\dagger} $ (đọc là \textit{A dagger} trong tiếng Anh). (\textit{Giả nghịch đảo (pseudo inverse) là trường hợp tổng quát của nghịch đảo khi ma trận không khả nghịch hoặc thậm chí không vuông. Trong khuôn khổ bài viết này, tôi xin phép được lược bỏ phần này, nếu các bạn thực sự quan tâm, tôi sẽ viết một bài khác chỉ nói về giả nghịch đảo. Xem thêm: \href{http://www.sci.utah.edu/~gerig/CS6640-F2012/Materials/pseudoinverse-cis61009sl10.pdf}{Least Squares, Pseudo-Inverses, PCA \& SVD}.})
 
Với khái niệm giả nghịch đảo, điểm tối ưu của bài toán Linear Regression có dạng: 
\begin{equation}
\label{eqn:linearregression_solution}
\mathbf{w} = \mathbf{A}^{\dagger}\mathbf{b} = (\mathbf{\bar{X}}^T\mathbf{\bar{X}})^{\dagger} \mathbf{\bar{X}}^T\mathbf{y} 
\end{equation} 
 
 
 
\section{Ví dụ trên Python}
 
 
\subsection{Bài toán}
 
Trong phần này, tôi sẽ chọn một ví dụ đơn giản về việc giải bài toán Linear Regression trong Python. Tôi cũng sẽ so sánh nghiệm của bài toán khi giải theo phương trình \eqref{eqn:linearregression_solution} và nghiệm tìm được khi dùng thư viện \href{http://scikit-learn.org/stable/}{scikit-learn} của Python. (\textit{Đây là thư viện Machine Learning được sử dụng rộng rãi trong Python}). Trong ví dụ này, dữ liệu đầu vào chỉ có 1 giá trị (1 chiều) để thuận tiện cho việc minh hoạ trong mặt phẳng.  
 
Chúng ta có 1 bảng dữ liệu về chiều cao và cân nặng của 15 người như trong Bảng \ref{tab:height_weight}. 
 
 
\begin{table}[h!]
\centering
\caption{Bảng dữ liệu về chiều cao và cân nặng của 15 người}
\label{tab:height_weight}
\begin{tabular}{|c|c|c|c|}
\hline
\textbf{Chiều cao (cm)} & \textbf{Cân nặng (kg)} & \textbf{Chiều cao (cm)} & \textbf{Cân nặng (kg)} \\ \hline
147                     & 49                     & 168                     & 60                     \\ \hline
150                     & 50                     & 170                     & 72                     \\ \hline
153                     & 51                     & 173                     & 63                     \\ \hline
155                     & 52                     & 175                     & 64                     \\ \hline
158                     & 54                     & 178                     & 66                     \\ \hline
160                     & 56                     & 180                     & 67                     \\ \hline
163                     & 58                     & 183                     & 68                     \\ \hline
165                     & 59                     &                         &                        \\ \hline
\end{tabular}
\end{table}

Bài toán đặt ra là: liệu có thể dự đoán cân nặng của một người dựa vào chiều cao của họ không? (\textit{Trên thực tế, tất nhiên là không, vì cân nặng còn phụ thuộc vào nhiều yếu tố khác nữa, thể tích chẳng hạn}). Vì blog này nói về các thuật toán Machine Learning đơn giản nên tôi sẽ giả sử rằng chúng ta có thể dự đoán được. 
 
Chúng ta có thể thấy là cân nặng sẽ tỉ lệ thuận với chiều cao (càng cao càng nặng), nên có thể sử dụng Linear Regression model cho việc dự đoán này. Để kiểm tra độ chính xác của model tìm được, chúng ta sẽ giữ lại cột 155 và 160 cm để kiểm thử, các cột còn lại được sử dụng để huấn luyện (train) model. 
 
 
\subsection{Hiển thị dữ liệu trên đồ thị}
Trước tiên, chúng ta cần có hai thư viện \href{http://www.numpy.org/}{numpy} cho đại số tuyến tính và \href{http://matplotlib.org/}{matplotlib} cho việc vẽ hình.  
 
 
\begin{lstlisting}[language=Python]
# To support both python 2 and python 3 
from __future__ import division, print_function, unicode_literals 
import numpy as np  
import matplotlib.pyplot as plt 
\end{lstlisting}
 
Tiếp theo, chúng ta khai báo và biểu diễn dữ liệu trên một đồ thị. 
 
 
\begin{lstlisting}[language=Python]
# height (cm) 
X = np.array([[147, 150, 153, 158, 163, 165, 168, 170, 173, 175, 178, 180, 183]]).T 
# weight (kg) 
y = np.array([[ 49, 50, 51,  54, 58, 59, 60, 62, 63, 64, 66, 67, 68]]).T 
# Visualize data  
plt.plot(X, y, 'ro') 
plt.axis([140, 190, 45, 75]) 
plt.xlabel('Height (cm)') 
plt.ylabel('Weight (kg)') 
plt.show() 
\end{lstlisting}
 
 
% <div class="imgcap"> 
% <img src ="/assets/LR/output_3_0.png" align = "center"> 
% </div> 
 
 \begin{figure}
 	\centering	
 	\includegraphics[width = .6\textwidth]{../LR/output_3_0.png}
 	\caption{ Phân bố của dữ liệu trong ví dụ có dạng đường thẳng}
 	\label{fig:linearregression_distribution}
 \end{figure}
 
Từ đồ thị trong Hình \ref{fig:linearregression_distribution} ta thấy rằng dữ liệu được sắp xếp gần như theo 1 đường thẳng, vậy mô hình Linear Regression nhiều khả năng sẽ cho kết quả tốt: 
 
(cân nặng) = \pythoninline{w\_1}*(chiều cao) + \pythoninline{w\_0} 
 
 
\subsection{Nghiệm theo công thức}
 
Tiếp theo, chúng ta sẽ tính toán các hệ số \pythoninline{w_1} và \pythoninline{w_0} dựa vào công thức $(5)$. Chú ý: giả nghịch đảo của một ma trận \pythoninline{A} trong Python sẽ được tính bằng \pythoninline{numpy.linalg.pinv(A)}, \pythoninline{pinv} là từ viết tắt của \textit{pseudo inverse}. 
 
\begin{lstlisting}[language=Python]
# Building Xbar  
one = np.ones((X.shape[0], 1)) 
Xbar = np.concatenate((one, X), axis = 1) 
 
# Calculating weights of the fitting line  
A = np.dot(Xbar.T, Xbar) 
b = np.dot(Xbar.T, y) 
w = np.dot(np.linalg.pinv(A), b) 
print('w = ', w) 
# Preparing the fitting line  
w_0 = w[0][0] 
w_1 = w[1][0] 
x0 = np.linspace(145, 185, 2) 
y0 = w_0 + w_1*x0 
 
# Drawing the fitting line  
plt.plot(X.T, y.T, 'ro')     # data  
plt.plot(x0, y0)               # the fitting line 
plt.axis([140, 190, 45, 75]) 
plt.xlabel('Height (cm)') 
plt.ylabel('Weight (kg)') 
plt.show() 
 
\end{lstlisting}
 
\begin{lstlisting}[language=Python]
w =  [[-33.73541021] 
 [  0.55920496]] 
\end{lstlisting}
 
 


\begin{figure}
	\centering
	\includegraphics[width = .6\textwidth]{../LR/output_5_1.png}
	\caption{Minh họa nghiệm của ví dụ}
\end{figure}
 
Từ đồ thị bên trên ta thấy rằng các điểm dữ liệu màu đỏ nằm khá gần đường thẳng dự đoán màu xanh. Vậy mô hình Linear Regression hoạt động tốt với tập dữ liệu \textit{training}. Bây giờ, chúng ta sử dụng mô hình này để dự đoán cân nặng của hai người có chiều cao 155 và 160 cm mà chúng ta đã không dùng khi tính toán nghiệm. 
 
 
\begin{lstlisting}[language=Python]
y1 = w_1*155 + w_0 
y2 = w_1*160 + w_0 
 
print( 'Predict weight of person 155 cm: %.2f (kg), real number: 52 (kg)'  %(y1) ) 
print( 'Predict weight of person 160 cm: %.2f (kg), real number: 56 (kg)'  %(y2) ) 
\end{lstlisting}
 

\begin{lstlisting}[language=Python]
Predict weight of person 155 cm: 52.94 (kg), real number: 52 (kg) 
Predict weight of person 160 cm: 55.74 (kg), real number: 56 (kg) 
\end{lstlisting}
 
 
Chúng ta thấy rằng kết quả dự đoán khá gần với số liệu thực tế. 
 
 
\subsection{Nghiệm theo thư viện scikit-learn}
 
Tiếp theo, chúng ta sẽ sử dụng thư viện scikit-learn của Python để tìm nghiệm.  
 
 
\begin{lstlisting}[language=Python]
from sklearn import datasets, linear_model 
 
# fit the model by Linear Regression 
regr = linear_model.LinearRegression(fit_intercept=False) # fit_intercept = False for calculating the bias 
regr.fit(Xbar, y) 
 
# Compare two results 
print( 'Solution found by scikit-learn  : ', regr.coef_ ) 
print( 'Solution found by (5): ', w.T) 
\end{lstlisting}
 
\begin{lstlisting}[language=Python]
    Solution found by scikit-learn  :  [[  -33.73541021 0.55920496]] 
    Solution found by (5):  [[  -33.73541021 0.55920496 ]] 
\end{lstlisting}
 
Chúng ta thấy rằng hai kết quả thu được như nhau! (\textit{Nghĩa là tôi đã không mắc lỗi nào trong cách tìm nghiệm ở phần trên}) 
 
\href{https://github.com/tiepvupsu/tiepvupsu.github.io/blob/master/assets/LR/LR.ipynb}{Source code Jupyter Notebook cho bài này.} 
 
 
 
\section{Thảo luận}
 
 
\subsection{Các bài toán có thể giải bằng Linear Regression}
Hàm số $y \approx f(\mathbf{x})= \mathbf{w}^T\mathbf{x}$ là một hàm tuyến tính theo cả $ \mathbf{w}$ và $\mathbf{x}$. Trên thực tế, Linear Regression có thể áp dụng cho các mô hình chỉ cần tuyến tính theo $\mathbf{w}$. Ví dụ: 
$$ 
y \approx w_1 x_1 + w_2 x_2 + w_3 x_1^2 +  w_4 \sin(x_2) + w_5 x_1x_2 + w_0 
$$ 
là một hàm tuyến tính theo $\mathbf{w}$ và vì vậy cũng có thể được giải bằng Linear Regression. Với mỗi dữ liệu đầu vào $\mathbf{x}=[x_1; x_2] $, chúng ta tính toán dữ liệu mới $\tilde{\mathbf{x}} = [x_1, x_2, x_1^2, \sin(x_2), x_1x_2]$ (đọc là \textit{x tilde} trong tiếng Anh) rồi áp dụng Linear Regression với dữ liệu mới này.  
 
Xem thêm ví dụ về \href{http://www.varsitytutors.com/hotmath/hotmath_help/topics/quadratic-regression}{Quadratic Regression} (Hồi Quy Bậc Hai). 
% <div class="imgcap"> 
% <div > 
%     <img src ="http://www.varsitytutors.com/assets/vt-hotmath-legacy/hotmath_help/topics/quadratic-regression/f-qr-1-1.gif"  width = "300"></a> 
% </div> 
% <div class="thecap"> Quadratic Regression (Nguồn: <a href = "http://www.varsitytutors.com/hotmath/hotmath_help/topics/quadratic-regression"> Quadratic Regression</a>) <br></div> 
% </div> 
 
 \begin{figure}
   \centering
   \includegraphics[width = .4\textwidth]{../LR/qr.png}
   \caption{Quadratic Regression (Nguồn: \href{http://www.varsitytutors.com/hotmath/hotmath_help/topics/quadratic-regression}{Quadratic Regression})}
   \label{fig:linearregression_qr}
 \end{figure}
 
\subsection{Hạn chế của Linear Regression}
 
Hạn chế đầu tiên của Linear Regression là nó rất \textbf{nhạy cảm với nhiễu} (sensitive to noise). Trong ví dụ về mối quan hệ giữa chiều cao và cân nặng bên trên, nếu có chỉ 
một cặp dữ liệu \textit{nhiễu} (150 cm, 90kg) thì kết quả sẽ sai khác đi rất nhiều (Xem Hình \ref{fig:linearregression_noise}). 
% <div class="imgcap"> 
% <img src ="/assets/LR/output_13_1.png" align = "center"> 
% </div> 
 
 \begin{figure}
 	\centering
 	\includegraphics[width = .6\textwidth]{../LR/output_13_1.png}
 	\caption{Linear Regression rất nhạy cảm với nhiễu}
 	\label{fig:linearregression_noise}
 \end{figure}
Vì vậy, trước khi thực hiện Linear Regression, các nhiễu (\textit{outlier}) cần 
 phải được loại bỏ. Bước này được gọi là tiền xử lý (pre-processing). 
 
Hạn chế thứ hai của Linear Regression là nó \textbf{không biễu diễn được các mô hình phức tạp}. Mặc dù trong phần trên, chúng ta thấy rằng phương pháp này có thể được áp dụng nếu quan hệ giữa \textit{outcome} và \textit{input} không nhất thiết phải là tuyến tính, nhưng mối quan hệ này vẫn đơn giản nhiều so với các mô hình thực tế. Hơn nữa, chúng ta sẽ tự hỏi: làm thế nào để xác định được các hàm $x_1^2, \sin(x_2), x_1x_2$ như ở trên?! 
 
 
\subsection{Các phương pháp tối ưu}
Linear Regression là một mô hình đơn giản, lời giải cho phương trình đạo hàm bằng 0 cũng khá đơn giản. \textit{Trong hầu hết các trường hợp, chúng ta không thể giải được phương trình đạo hàm bằng 0.} 
 
Nhưng có một điều chúng ta nên nhớ, \textbf{còn tính được đạo hàm là còn có cơ hội}. 
 
 
 
 
\section{Tài liệu tham khảo}
 \begin{enumerate}
 	\item  \href{https://en.wikipedia.org/wiki/Linear_regression}{Linear Regression - Wikipedia}

 	\item  \href{http://machinelearningmastery.com/simple-linear-regression-tutorial-for-machine-learning/}{Simple Linear Regression Tutorial for Machine Learning} 

 	\item  \href{http://www.sci.utah.edu/~gerig/CS6640-F2012/Materials/pseudoinverse-cis61009sl10.pdf}{Least Squares, Pseudo-Inverses, PCA \& SVD} 
 \end{enumerate}



\begin{lstlisting}[language=Python, caption=Python example]
import numpy as np
 
def incmatrix(genl1,genl2):
    m = len(genl1)
    n = len(genl2)
    M = None #to become the incidence matrix
    VT = np.zeros((n*m,1), int)  #dummy variable
 
    #compute the bitwise xor matrix
    M1 = bitxormatrix(genl1)
    M2 = np.triu(bitxormatrix(genl2),1) 
 
    for i in range(m-1):
        for j in range(i+1, m):
            [r,c] = np.where(M2 == M1[i,j])
            for k in range(len(r)):
                VT[(i)*n + r[k]] = 1;
                VT[(i)*n + c[k]] = 1;
                VT[(j)*n + r[k]] = 1;
                VT[(j)*n + c[k]] = 1;
 
                if M is None:
                    M = np.copy(VT)
                else:
                    M = np.concatenate((M, VT), 1)
 
                VT = np.zeros((n*m,1), int)
 
    return M
\end{lstlisting}


%\include{chapter}
%\appendix
%\include{appendix}

\backmatter%%%%%%%%%%%%%%%%%%%%%%%%%%%%%%%%%%%%%%%%%%%%%%%%%%%%%%%
% \include{solutions}
% %%%%%%%%%%%%%%%%%%%%%%%% referenc.tex %%%%%%%%%%%%%%%%%%%%%%%%%%%%%%
% sample references
% "computer science"
%
% Use this file as a template for your own input.
%
%%%%%%%%%%%%%%%%%%%%%%%% Springer-Verlag %%%%%%%%%%%%%%%%%%%%%%%%%%

%
% BibTeX users please use
% \bibliographystyle{}
% \bibliography{}
%
% Non-BibTeX users please use
\begin{thebibliography}{99.}
%
% and use \bibitem to create references.
%
% Use the following syntax and markup for your references
%
% Monographs
\bibitem{monograph} Kajan E (2002)
Information technology encyclopedia and acronyms. Springer, Berlin
Heidelberg New York

% Contributed Works
\bibitem{contribution} Broy M (2002) Software engineering -- From
auxiliary to key technologies. In: Broy M, Denert E (eds)
Software Pioneers. Springer, Berlin Heidelberg New York

% Journal
\bibitem{journal} Che M, Grellmann W, Seidler S (1997)
Appl Polym Sci 64:1079--1090

% Theses
\bibitem{thesis} Ross DW (1977) Lysosomes and storage diseases. MA
Thesis, Columbia University, New York

\end{thebibliography}

\printindex

%%%%%%%%%%%%%%%%%%%%%%%%%%%%%%%%%%%%%%%%%%%%%%%%%%%%%%%%%%%%%%%%%%%%%%

\end{document}






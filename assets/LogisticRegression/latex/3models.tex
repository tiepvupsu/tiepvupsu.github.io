\documentclass{standalone}
\usepackage{amsmath}
\usepackage{tikz}
\renewcommand{\familydefault}{\sfdefault}


\begin{document}
\pagestyle{empty}
% \input{definitions}
\def\layersep{2.5cm}
\tikzset{dist/.style={path picture= {
    \begin{scope}[x=1pt,y=10pt]
        \def\a{5}
      \draw [blue] (-\a, -.5) --  (\a, .5);
    \end{scope}
    }
  }
}

\tikzset{distt/.style={path picture= {
    \begin{scope}[x=1pt,y=10pt]
        \def\a{5}
      \draw [blue] (-\a, -.5) -- (0, -.5) -- (0, .5) -- (\a, .5);
    \end{scope}
    }
  }
}

\tikzset{disttt/.style={path picture= {
    \begin{scope}[x=1pt,y=10pt]
        \def\a{5}
      \draw [blue] (-\a, -.5) -- (0, -.5) -- (0, .5) -- (\a, .5);
    \end{scope}
    }
  }
}

\tikzset{disttt/.style={path picture= {
    \begin{scope}[x=1pt,y=10pt]
        \def\a{5}
      % \draw [blue] (-\a, -.5) -- (0, -.5) -- (0, .5) -- (\a, .5);
      % \draw[domain=-.5:.5, smooth, variable=\x,blue] plot ({\x},{1/(1 + exp(-\x))}
      % \addplot[color=red]{1/(1+exp(-x))-2};
      \draw [blue] plot[domain=-6:6] (\x,{1/(1 + exp(-\x))-0.5});
    \end{scope}
    }
  }
}



\tikzstyle{every pin edge}=[shorten <=1pt]
\tikzstyle{neuron}=[circle,fill=black!25,minimum size=17pt,inner sep=0pt]
\tikzstyle{input neuron}=[neuron, thick, draw = black, fill=green!40]
\tikzstyle{output neuron}=[neuron,thick, draw = black, fill=red!40]
\tikzstyle{linear neuron}=[output neuron, dist]
\tikzstyle{sign neuron}=[output neuron, distt]
\tikzstyle{sigmoid neuron}=[output neuron, disttt]
\tikzstyle{hidden neuron}=[neuron, thick, draw = black, fill=blue!40]
\tikzstyle{annot} = [text width=4em, text centered]
\begin{tikzpicture}[shorten >=1pt,draw=black,thick, node distance=\layersep, >= stealth]

    % Draw the input layer nodes
    \foreach \name / \y in {1,...,4}
    % This is the same as writing \foreach \name / \y in {1/1,2/2,3/3,4/4}
        \node[input neuron] (I-\name) at (0,-\y) {};


    % Draw the output layer nodes
    \foreach \name / \y in {3}
        \path[yshift=0.5cm]
            node[linear neuron ] (O-\name) at (\layersep,-\y cm) {};
    % hidden layer.
    \foreach \source in {1,...,4}
        \foreach \dest in {3}
            % \path (I-\source) edge (H-\dest);
            \draw [thick, ->](I-\source) -- (O-\dest);
    \node at (1, -5) {Linear Regression};
\end{tikzpicture}
% sign 
\quad \quad
\begin{tikzpicture}[shorten >=1pt,draw=black,thick, node distance=\layersep, >= stealth]

    % Draw the input layer nodes
    \foreach \name / \y in {1,...,4}
    % This is the same as writing \foreach \name / \y in {1/1,2/2,3/3,4/4}
        \node[input neuron] (I-\name) at (0,-\y) {};


    % Draw the output layer nodes
    \foreach \name / \y in {3}
        \path[yshift=0.5cm]
            node[sign neuron ] (O-\name) at (\layersep,-\y cm) {};
    % hidden layer.
    \foreach \source in {1,...,4}
        \foreach \dest in {3}
            % \path (I-\source) edge (H-\dest);
            \draw [thick, ->](I-\source) -- (O-\dest);
    \node at (1, -5) {PLA};
\end{tikzpicture}
\quad \quad
\begin{tikzpicture}[shorten >=1pt,draw=black,thick, node distance=\layersep, >= stealth]

    % Draw the input layer nodes
    \foreach \name / \y in {1,...,4}
    % This is the same as writing \foreach \name / \y in {1/1,2/2,3/3,4/4}
        \node[input neuron] (I-\name) at (0,-\y) {};


    % Draw the output layer nodes
    \foreach \name / \y in {3}
        \path[yshift=0.5cm]
            node[sigmoid neuron ] (O-\name) at (\layersep,-\y cm) {};
    % hidden layer.
    \foreach \source in {1,...,4}
        \foreach \dest in {3}
            % \path (I-\source) edge (H-\dest);
            \draw [thick, ->](I-\source) -- (O-\dest);
    \node at (1, -5) {Logistic Regression};
\end{tikzpicture}

% End of code
\end{document}
